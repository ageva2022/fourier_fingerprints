%%%%%%%%%%%%%%%%%%%%%%%%
% DO NOT CHANGE HERE
%%%%%%%%%%%%%%%%%%%%%%%%
\documentclass[letter]{amsart}
\renewcommand{\thesubsection}{\Alph{subsection}}
\def\doubleunderline#1{\underline{\underline{#1}}}
\newcommand\tab[1][1cm]{\hspace*{#1}}
\usepackage{titlesec}
\usepackage{textcomp}
\titleformat{\subsection}[frame]
{\normalfont} {} {2pt} {\normalsize\bfseries\filright\thesubsection.\quad}
\usepackage[english]{babel}
\usepackage[utf8]{inputenc}
\usepackage{enumerate}
\usepackage{amsmath,amsfonts, amssymb}
\usepackage{graphicx}
\usepackage[table,xcdraw]{xcolor}
\usepackage{tikz}
\usetikzlibrary{positioning,shapes,arrows}
\usepackage[margin=1in]{geometry}
\titleformat {\section}
  {\normalfont \Large \bfseries \centering}{\thesection}{1em}{}
\usepackage{enumitem}
\usepackage{mathrsfs}
\usepackage{caption}
\usepackage{hyperref}
\usepackage{subcaption}
\usepackage{float}
\usepackage[fontsize=12pt]{scrextend}
\restylefloat{table}
\usepackage{mathtools}
\newcommand{\rr}{\mathbb{R}}
\newcommand{\nn}{\mathbb{N}}
\newcommand{\qq}{\mathbb{Q}}
\newcommand{\dd}{$D$ }
\newcommand{\intt}{int \text{ }}
\newcommand{\bd}{bd \text{ }}
\newcommand{\nbd}{nbd \text{ }}
\newcommand{\cl}{cl \text{ }}
\newcommand{\me}{\mathrm{e}}
\newcommand\mypound{\protect\scalebox{0.8}{\protect\raisebox{0.4ex}{\#}}}
\definecolor{UMassMaroon}{RGB}{136,28,28}
\setcounter{MaxMatrixCols}{20}

%%%%%%%%%%%%%%%%%
% CHANGE HERE
%%%%%%%%%%%%%%%%%
\newcommand{\StudentName}{Harold Thidemann. Adi Geva, Dhruvi LASTNAME}

%%%%%%%%%%%%%%%%%%%%%%%%
% DO NOT CHANGE HERE
%%%%%%%%%%%%%%%%%%%%%%%%
\title[Math 552: Applications Of Scientific Computing $\mid$ Project ]{Math 552: Applications Of Scientific Computing\\Project}
\author[\StudentName]{\StudentName}

\begin{document}
\maketitle

%%%%%%%%%%%%%%%%%
% CHANGE HERE
%%%%%%%%%%%%%%%%%

\section*{Section 1}

We should probably start here, we are NOT making it double columned, and there are examples on how to insert images later on.

\newpage

\section*{Section 2.1}
If a problem has multiple parts, please put a \verb!\newpage! between each subpart.  For example, like this.
\newpage

\section*{Section 2.2}
And like this.
\newpage

\section*{Section 2.3}
And like this.
\newpage

\section*{Section 3}
Similarly, if a problem (or a subpart) ends up taking more than 1 page, please put (Continued) in the \verb!\section*! for that page.  You can also use \verb!\newpage! to choose where (in your work) the current page ends and where the next page begins.
\newpage

\section*{Section 3 (Continued)}
Like this.
\newpage

\section*{Example Proof}
\noindent\textbf{Want to Prove:} $x+y=x+z \ni x,y,z \in \mathcal{Z} \implies y=z$.\vspace{0.5em}\\
$\begin{aligned}
x+y &= x + z\\
(-x)+(x+y) &= (-x)+(x+z)\\
((-x)+x)+y&=((-x)+x)+z\\
0+y &= 0+z\\
y &= z \qed
\end{aligned}$
\text{ }\vspace{2em}\\
\textbf{\LaTeX\ Pro Tip}: They key to the \texttt{aligned} environment is to use the \& operator as what everything aligns to.  Note that, above, we always have it as \texttt{\&=}, meaning that each line has the \texttt{=} sign aligned to the same spot.  This makes the proof much easier to read.\vspace{1em}\\
\textbf{Math Pro Tip}: The $\qed$ symbol, called \href{https://en.wikipedia.org/wiki/Q.E.D.}{Q.E.D.}, is often used to signal the end of a proof.
\newpage

\section*{Example Math Work}
$\begin{aligned}
(1 + 2 + 3 + 4 + 5) \cdot 1000 &= (3+3+4+5) \cdot 1000\\
&= (6+4+5) \cdot 1000\\
&= (10+5) \cdot 1000\\
&= (15) \cdot 1000\\
1 + 2 + 3 + 4 + 5 &= \doubleunderline{15000}\\
\end{aligned}$
\text{ }\vspace{2em}\\
\textbf{\LaTeX\ Pro Tip}: Feel free to use the \verb!\doubleunderline! command to double-underline your final answer, which can make your work easier to read.
\newpage

\section*{Example Table}
\textbf{\LaTeX\ Pro Tip}: Use \href{https://tablesgenerator.com/}{tablesgenerator.com} to save \textit{a lot} of time when making tables.  You can even copy/paste from Excel or import a CSV!
\begin{table}[H]
\begin{tabular}{|c|c|c|c|c|}
\hline
\rowcolor[HTML]{333333} 
{\color[HTML]{FFFFFF} \textbf{Column 1}} &
  {\color[HTML]{FFFFFF} \textbf{Column 2}} &
  {\color[HTML]{FFFFFF} \textbf{Column 3}} &
  {\color[HTML]{FFFFFF} \textbf{Column 4}} &
  {\color[HTML]{FFFFFF} \textbf{Column 5}} \\ \hline
 &  &  &  &  \\ \hline
 &  &  &  &  \\ \hline
 &  &  &  &  \\ \hline
 &  &  &  &  \\ \hline
 &  &  &  &  \\ \hline
\end{tabular}
\caption{Table Caption Here}
\end{table}
\newpage

\section*{Using an Uploaded Photo}
\textbf{\LaTeX\ Pro Tip}: Avoid using underscores in file names to make your life easier.  \href{https://tex.stackexchange.com/questions/58689/how-to-use-an-underscore-in-a-filename}{There are ways around this} but it's likely easier to just rename a file.
\begin{figure}[H]
    \centering
    \includegraphics[width=0.5\textwidth]{cute-kitten.jpg}
    \caption{Example Image A}
\end{figure}
\newpage

\section*{Single Figure}
\textbf{\LaTeX\ Pro Tip}: To remove the image caption, comment out the \verb!\caption! line.
\begin{figure}[H]
    \centering
    \includegraphics[width=0.5\textwidth]{example-image-a}
    \caption{Example Image A}
\end{figure}
\newpage

\section*{Two Side-By-Side Figures}
\begin{minipage}[H]{0.5\textwidth}
    \centering
    \includegraphics[width=0.8\textwidth]{example-image-a}
    \captionof{figure}{Example Image A}
\end{minipage}
\begin{minipage}[H]{0.5\textwidth}
    \centering
    \includegraphics[width=0.8\textwidth]{example-image-b}
    \captionof{figure}{Example Image B}
\end{minipage}
\newpage

\section*{Three Side-By-Side Figures}
\begin{minipage}[H]{0.33\textwidth}
    \centering
    \includegraphics[width=0.95\textwidth]{example-image-a}
    \captionof{figure}{Example Image A}
\end{minipage}
\begin{minipage}[H]{0.33\textwidth}
    \centering
    \includegraphics[width=0.95\textwidth]{example-image-b}
    \captionof{figure}{Example Image B}
\end{minipage}
\begin{minipage}[H]{0.33\textwidth}
    \centering
    \includegraphics[width=0.95\textwidth]{example-image-c}
    \captionof{figure}{Example Image C}
\end{minipage}
\newpage

\section*{Side-By-Side Figure and Table}
\begin{minipage}[H]{0.5\textwidth}
    \centering
    \includegraphics[width=0.8\textwidth]{example-image-a}
    \captionof{figure}{Example Image A}
\end{minipage}
\begin{minipage}[H]{0.5\textwidth}
\resizebox{\textwidth}{!}{%
\begin{tabular}{|c|c|c|c|c|}
\hline
\rowcolor[HTML]{333333} 
{\color[HTML]{FFFFFF} \textbf{Column 1}} &
  {\color[HTML]{FFFFFF} \textbf{Column 2}} &
  {\color[HTML]{FFFFFF} \textbf{Column 3}} &
  {\color[HTML]{FFFFFF} \textbf{Column 4}} &
  {\color[HTML]{FFFFFF} \textbf{Column 5}} \\ \hline
 &  &  &  &  \\ \hline
 &  &  &  &  \\ \hline
 &  &  &  &  \\ \hline
 &  &  &  &  \\ \hline
 &  &  &  &  \\ \hline
\end{tabular}%
}
\captionof{table}{Table Caption}
\end{minipage}
\newpage

\end{document}